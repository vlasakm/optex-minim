\fontfam[lm]
\load[tikz]
\load[minim-mp]

%\tracingall
%\pdfcompresslevel=0
%\pdfobjcompresslevel=0

\directlua{
local node_id  = node.id
local glyph_id = node_id("glyph")
local rule_id  = node_id("rule")
local glue_id  = node_id("glue")
local hlist_id = node_id("hlist")
local vlist_id = node_id("vlist")
local disc_id  = node_id("disc")

local direct       = node.direct
local todirect     = direct.todirect
local tonode       = direct.tonode
local getfield     = direct.getfield
local setfield     = direct.setfield
local getlist      = direct.getlist
local setlist      = direct.setlist
local getleader    = direct.getleader
local getattribute = direct.get_attribute
local insertbefore = direct.insert_before
local copy         = direct.copy
local traverse     = direct.traverse

local token_getmacro = token.get_macro

local pdfliteral = optex.directpdfliteral

function register_pre_shipout_injector(name, attribute, namespace, default)
    local current
    local default = default or 0
    local function injector(head)
        for n, id, subtype in traverse(head) do
            if id == hlist_id or id == vlist_id then
                % nested list, just recurse
                setlist(n, injector(getlist(n)))
            elseif id == disc_id then
                % only replace part is interesting at this point
                local replace = getfield(n, "replace")
                if replace then
                    setfield(n, "replace", injector(replace))
                end
            elseif id == glyph_id or id == rule_id
                    or (id == glue_id and getleader(n)) then
                local new = getattribute(n, attribute) or 0
                if new ~= current then
                    local literal = token_getmacro(namespace..new)
                    head = insertbefore(head, n, pdfliteral(literal))
                    current = new
                end
            end
        end
        return head
    end

    callback.add_to_callback("pre_shipout_filter", function(list)
        current = default
        return tonode(injector(todirect(list)))
    end, name)
end
}


\newattribute \transpattr
\newcount \transpcnt \transpcnt=1 % allocations start at 1
\def\transparency#1{\inittransparency \transpattr=
   \ifcsname transp::#1\endcsname \lastnamedcs\relax \else
      \transpcnt
      \sxdef{transp::#1}{\the\transpcnt}%
      \sxdef{transp:\the\transpcnt}{/Tr\the\transpcnt\space gs}%
      \addextgstate{Tr\the\transpcnt}{<</ca #1 /CA #1>>}%
      \incr \transpcnt
   \fi
}
\addto\_resetcolor{\transpattr=-"7FFFFFFF }
% Transparency of "1" is the default
\sdef{transp::1}{0}
\sdef{transp:0}{/Tr0 gs}
\def\inittransparency{%
   \addextgstate{Tr0}{<</ca 1 /CA 1>>}%
   \glet\inittransparency=\relax
}
\directlua{
register_pre_shipout_injector("transp", registernumber("transpattr"), "transp:")
}

\newattribute \fntoutattr
\newcount \fntoutcnt \fntoutcnt=1 % allocations start at 1
\def\outlinefont#1{\fntoutattr=
   \ifcsname fntout::#1\endcsname \lastnamedcs\relax \else
      \fntoutcnt
      \sxdef{fntout::#1}{\the\fntoutcnt}%
      \sxdef{fntout:\the\fntoutcnt}{#1 w 1 Tr}%
      \incr \fntoutcnt
   \fi
}
\addto\_resetcolor{\fntoutattr=-"7FFFFFFF }
\sdef{fntout:0}{0 w 0 Tr}
\directlua{
register_pre_shipout_injector("fntout", registernumber("fntoutattr"), "fntout:")
}

a{\transparency{.5}b{\transparency{.2}c}b}a

{\Red a{A\outlinefont{.1}B\pdfliteral{0 1 0 RG}\outlinefont{.3}B}a}

\inoval[\shadow=3]{abc}

% From PGF manual, section 23.2 "Specifying a Uniform Opacity"
\tikzpicture[line width=1ex]
\draw (0,0) -- (3,1);
%\filldraw [fill=yellow!80!black,draw opacity=0.5] (1,0) rectangle (2,1);
\filldraw [fill=yellow!80!black,draw opacity=0.5] (1,0) rectangle (2,1);
\endtikzpicture

% From PGF manual, section 17.13.2 "Referencing the Current Page Node - Absolute Positioning"
\tikzpicture [remember picture,overlay]
\draw [line width=1mm,opacity=.25] (current page.center) circle (3cm);
\endtikzpicture

% From PGF manual, section 15.10 "Doing Multiple Actions on a Path"
\usetikzlibrary {fadings,patterns}
\tikzpicture
  [
    % Define an interesting style
      button/.style={
      % First preaction: Fuzzy shadow
      preaction={fill=black,path fading=circle with fuzzy edge 20 percent,
                 opacity=.5,transform canvas={xshift=1mm,yshift=-1mm}},
      % Second preaction: Background pattern
      preaction={pattern=#1,
                 path fading=circle with fuzzy edge 15 percent},
      % Third preaction: Make background shiny
      preaction={top color=white,
                 bottom color=black!50,
                 shading angle=45,
                 path fading=circle with fuzzy edge 15 percent,
                 opacity=0.2},
      % Fourth preaction: Make edge especially shiny
      preaction={path fading=fuzzy ring 15 percent,
                 top color=black!5,
                 bottom color=black!80,
                 shading angle=45},
      inner sep=2ex
    },
    button/.default=horizontal lines light blue,
    circle
  ]

\draw [help lines] (0,0) grid (4,3);

\node [button] at (2.2,1) {\typosize[24.88/] Big};
\node [button=crosshatch dots light steel blue,
       text=white] at (1,1.5) {Small};
\endtikzpicture


% From PGF manual, section 23.4.1 "Creating Fadings"
{\loggingall
\tikzfadingfrompicture[name=fade right with circle]
  \shade[left color=transparent!0,
         right color=transparent!100] (0,0) rectangle (2,2);
  \fill[transparent!50] (1,1) circle (0.7);
\endtikzfadingfrompicture
}

% Now we use the fading in another picture:
\tikzpicture
  % Background
  \fill [black!20] (-1.2,-1.2) rectangle (1.2,1.2);
  \pattern [pattern=checkerboard,pattern color=black!30]
                   (-1.2,-1.2) rectangle (1.2,1.2);

  \fill [path fading=fade right with circle,red] (-1,-1) rectangle (1,1);
\endtikzpicture

\directmetapost{
% define the pattern
picture letter; letter = maketext("a");
beginpattern(a)
    draw letter rotated 45;
    matrix = identity rotated 45;
endpattern(12pt,12pt);
% use the pattern
beginfig(1)
    fill fullcircle scaled 3cm withpattern(a) withcolor 3/4red;
    draw fullcircle scaled 3cm withpen pencircle scaled 1;
endfig;
}

\vfil\break
\slides
\slideshow

\sec A

\layers 3
{\pshow2 Second text.} {\pshow3 Third text.} {\pshow1 First text.}
\endlayers

* a \pg+
* b \pg+
* c \pg+
* d

\bye
